%%%%%%%%%%%%%%%%%%%%%%%%%%%%%%%%%%%%%%%%%%%%%%%%%%%%%%%%%%%%%%%%%%%%%
% LaTeX Template: Project Titlepage Modified (v 0.1) by rcx
%
% Original Source: http://www.howtotex.com
% Date: February 2014
% 
% This is a title page template which be used for articles & reports.
% 
% This is the modified version of the original Latex template from
% aforementioned website.
% 
%%%%%%%%%%%%%%%%%%%%%%%%%%%%%%%%%%%%%%%%%%%%%%%%%%%%%%%%%%%%%%%%%%%%%%

\documentclass[12pt]{report}
\usepackage[a4paper]{geometry}
\usepackage[myheadings]{fullpage}
\usepackage{fancyhdr}
\usepackage{lastpage}
\usepackage{graphicx, wrapfig, subcaption, setspace, booktabs}
\usepackage[T1]{fontenc}
\usepackage[font=small, labelfont=bf]{caption}
\usepackage{fourier}
\usepackage[protrusion=true, expansion=true]{microtype}
\usepackage{sectsty}
\usepackage{commath}
\usepackage{url, lipsum}
\usepackage[utf8]{inputenc}
\usepackage{hyperref}
\usepackage[english]{babel}
\usepackage{csquotes}
\usepackage[backend=bibtex]{biblatex}
\addbibresource{cite.bib}


\newcommand{\HRule}[1]{\rule{\linewidth}{#1}}
\onehalfspacing
\setcounter{tocdepth}{5}
\setcounter{secnumdepth}{5}

%-------------------------------------------------------------------------------
% HEADER & FOOTER
%-------------------------------------------------------------------------------
\pagestyle{fancy}
\fancyhf{}
\setlength\headheight{15pt}
\fancyhead[L]{Advanced problems in Computer Science}
\fancyhead[R]{Object detection in Deep learning}
\fancyfoot[R]{\center{Page \thepage\ of \pageref{LastPage}}}
%-------------------------------------------------------------------------------
% TITLE PAGE
%-------------------------------------------------------------------------------

\begin{document}

\title{ \normalsize \textsc{PROJECT REPORT\\
Course: Advanced problems in Computer Science}
		\\ [5.0cm]
		\HRule{0.5pt} \\
		\LARGE \textbf{\uppercase{OBJECT DETECTION IN DEEP LEARNING}} 
		\\ [0.25 cm]
		\large {And their applications in real life}
		\HRule{2pt} \\ [0.5 cm]
		\normalsize  \vspace*{5\baselineskip}}

\date{ 
    \large{Academic year: 2020 - 2021}
}

\author{
        Dung Nguyen Manh \\
    }

\newpage
\maketitle

%-------------------------------------------------------------------------------
% Section title formatting
\sectionfont{\scshape}
%-------------------------------------------------------------------------------

\tableofcontents
\newpage

%-------------------------------------------------------------------------------
% ABSTRACTION
%-------------------------------------------------------------------------------

\begin{abstract}
    What did I do in a nutshell?     
\end{abstract}

%-------------------------------------------------------------------------------
% INTRODUCTION
%-------------------------------------------------------------------------------

\chapter{Introduction}
\section{A brief history of object detection}
What is the problem?
Cite something stupid from Object Detection Survey\cite{zou2019object}

%-------------------------------------------------------------------------------
% MATERIAL AND METHODS
%-------------------------------------------------------------------------------

\chapter{Material and Methods}
\section{Material}
\section{Methods}

%-------------------------------------------------------------------------------
% RESULT
%-------------------------------------------------------------------------------

\chapter{Result}
\section{Summary}
\section{Future work}

%-------------------------------------------------------------------------------
% DISCUSSION
%-------------------------------------------------------------------------------

\chapter{Discussion}


%-------------------------------------------------------------------------------
% CONCLUSION
%-------------------------------------------------------------------------------

\chapter{Conclusion}

%-------------------------------------------------------------------------------
% REFERENCES
%-------------------------------------------------------------------------------
\newpage


\printbibliography[title=References]

\end{document}
